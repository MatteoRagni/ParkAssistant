\documentclass{llncs}

\usepackage{graphicx}
\usepackage[english]{babel}
\usepackage{float}
\usepackage[unicode=true,bookmarks=true,bookmarksnumbered=false,bookmarksopen=false,breaklinks=false,pdfborder={0 0 0},backref=false,colorlinks=false]{hyperref}
\hypersetup{pdftitle={Park Assistant}, pdfauthor={Matteo Ragni -
161822},pdfsubject={Homework 02}}

\usepackage{amsmath}

\usepackage{listings}
\usepackage{color}

\definecolor{dkgreen}{rgb}{0,0.6,0}
\definecolor{gray}{rgb}{0.5,0.5,0.5}
\definecolor{mauve}{rgb}{0.58,0,0.82}

\lstset{frame=tb,
  language=C++,
  aboveskip=3mm,
  belowskip=3mm,
  showstringspaces=false,
  columns=flexible,
  basicstyle={\small\ttfamily},
  numbers=none,
  numberstyle=\tiny\color{gray},
  keywordstyle=\color{blue},
  commentstyle=\color{dkgreen},
  stringstyle=\color{mauve},
  breaklines=true,
  breakatwhitespace=true
  tabsize=3
}

% Definitions of newcommand

% Insert a new figure in the document, inside a float.
\newcommand{\fig}[4]{
	% /fig{image file}{caption}{scaling factor}{label name} 
	% 1 - FILE IMAGE										
	% 2 - CAPTION OF THE IMAGE								
	% 3 - SCALING FACTOR									
	% 4 - LABEL NAME										
	\begin{figure}[H] \label{#4}
		\centering
		\includegraphics[keepaspectratio, scale=#3]{#1}
		\caption{#2}
	\end{figure}
}


